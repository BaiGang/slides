% !TEX program = XeLaTeX
% !TEX encoding = UTF-8 Unicode

%%%%%%%%%%%%%%%%%%%%%%%%%%%%%%%%%%%%%%%%%%%%%%%%
% Multi-Label classification on Apache Spark
%
% License:
% CC BY-NC-SA 3.0 (http://creativecommons.org/licenses/by-nc-sa/3.0/)
%
%%%%%%%%%%%%%%%%%%%%%%%%%%%%%%%%%%%%%%%%%%%%%%%%

%%%%%%%%%%%%%%%%%%%%%%%%%%%%%%%%%%%%%%%%%
% Beamer Presentation
% LaTeX Template
% Version 1.0 (10/11/12)
%
% This template has been downloaded from:
% http://www.LaTeXTemplates.com
%
% License:
% CC BY-NC-SA 3.0 (http://creativecommons.org/licenses/by-nc-sa/3.0/)
%
%%%%%%%%%%%%%%%%%%%%%%%%%%%%%%%%%%%%%%%%%

%----------------------------------------------------------------------------------------
%	PACKAGES AND THEMES
%----------------------------------------------------------------------------------------

\documentclass{beamer}

\mode<presentation> {

% The Beamer class comes with a number of default slide themes
% which change the colors and layouts of slides. Below this is a list
% of all the themes, uncomment each in turn to see what they look like.

%\usetheme{default}
%\usetheme{AnnArbor}
%\usetheme{Antibes}
%\usetheme{Bergen}
%\usetheme{Berkeley}
%\usetheme{Berlin}
%\usetheme{Boadilla}
%\usetheme{CambridgeUS}
%\usetheme{Copenhagen}
%\usetheme{Darmstadt}
%\usetheme{Dresden}
\usetheme{Frankfurt}
%\usetheme{Goettingen}
%\usetheme{Hannover}
%\usetheme{Ilmenau}
%\usetheme{JuanLesPins}
%\usetheme{Luebeck}
%\usetheme{Madrid}
%\usetheme{Malmoe}
%\usetheme{Marburg}
%\usetheme{Montpellier}
%\usetheme{PaloAlto}
%\usetheme{Pittsburgh}
%\usetheme{Rochester}
%\usetheme{Singapore}
%\usetheme{Szeged}
%\usetheme{Warsaw}

% As well as themes, the Beamer class has a number of color themes
% for any slide theme. Uncomment each of these in turn to see how it
% changes the colors of your current slide theme.

%\usecolortheme{albatross}
%\usecolortheme{beaver}
%\usecolortheme{beetle}
%\usecolortheme{crane}
%\usecolortheme{dolphin}
%\usecolortheme{dove}
%\usecolortheme{fly}
%\usecolortheme{lily}
%\usecolortheme{orchid}
%\usecolortheme{rose}
%\usecolortheme{seagull}
%\usecolortheme{seahorse}
%\usecolortheme{whale}
%\usecolortheme{wolverine}

%\setbeamertemplate{footline} % To remove the footer line in all slides uncomment this line
\setbeamertemplate{headline} % To remove the header line in all slides uncomment this line
%\setbeamertemplate{footline}[page number] % To replace the footer line in all slides with a simple slide count uncomment this line

%\setbeamertemplate{navigation symbols}{} % To remove the navigation symbols from the bottom of all slides uncomment this line
}

\usepackage{graphicx} % Allows including images
\usepackage{booktabs} % Allows the use of \toprule, \midrule and \bottomrule in tables
\usepackage{algorithm2e}
\usepackage{fontspec}
\usepackage{xeCJK}

\setCJKmainfont[BoldFont=STHeiti,ItalicFont=STFangsong]{STKaiti}
\hypersetup{colorlinks=false}

%----------------------------------------------------------------------------------------
%	TITLE PAGE
%----------------------------------------------------------------------------------------

\title[MultiBoost]{Multi-Label Classification with Boosting \\ on Apache Spark} 

\author{
白刚  \href{http://weibo.com/baigang111}{@BaiGang-}
} % Your name
\institute[Sina] % Your institution as it will appear on the bottom of every slide, may be shorthand to save space
{
\textit{Sina Ad-Algo}
}
%\date{\today} % Date, can be changed to a custom date
\date{March 21, 2015.} % Date, can be changed to a custom date

\begin{document}

\begin{frame}
\titlepage % Print the title page as the first slide
\end{frame}

\begin{frame}
\frametitle{关于这一文档}
\begin{itemize}
\item TODO: 自我介绍
\item \LaTeX Beamer presentation. 
\item \url{https://github.com/baigang/slides}
\end{itemize}
\end{frame}

\begin{frame}
\frametitle{Overview} % Table of contents slide, comment this block out to remove it
\tableofcontents % Throughout your presentation, if you choose to use \section{} and \subsection{} commands, these will automatically be printed on this slide as an overview of your presentation
\end{frame}

%----------------------------------------------------------------------------------------
%	PRESENTATION SLIDES
%----------------------------------------------------------------------------------------

%------------------------------------------------

\section{背景}

\subsection{The problem}
\begin{frame}
\frametitle{背景:The problem}

  \begin{block}{Business}
   广告为媒体带来收益,同时也潜在的破坏用户体验。
  \end{block}

  \pause

  \begin{columns}[T]
    \begin{column}{.5\textwidth}
      \begin{center}
      \includegraphics[width=0.8\textwidth]{img/good_ad.png}
      \end{center}
      \begin{center}
      \includegraphics[width=0.15\textwidth]{img/emoji_smiley.png}
      \end{center}
    \end{column}
    \begin{column}{.5\textwidth}
      \begin{center}
      \includegraphics[width=0.8\textwidth]{img/bad_ad.png}
      \end{center}
      \begin{center}
      \includegraphics[width=0.15\textwidth]{img/emoji_unamused.png}
      \end{center}
    \end{column}
  \end{columns}

  \pause

  \begin{block}{}
   不同的广告有不同的受众。\\
   如何将广告投放给其对应的受众?
  \end{block}

\end{frame}

\subsection{User Profiling}

\begin{frame}
\frametitle{背景:User profiling}
\begin{figure}
\includegraphics[width=.6\linewidth]{img/target_market.png}
\end{figure}
\begin{block}{理解用户兴趣}
一个用户属于哪个人群,是哪些广告的潜在受众。
\end{block}
\end{frame}

\begin{frame}
\frametitle{背景: User profiling}
\begin{example}{用户兴趣}
  \begin{columns}

   \pause

    \begin{column}{0.333\textwidth}
      \begin{block}{User1}
      \begin{itemize}
        \item 财经-基金
        \item 财经-股票
        \item 房产-装修
        \item ...
      \end{itemize}
      \end{block}
    \end{column}

    \pause

    \begin{column}{0.333\textwidth}
      \begin{block}{User2}
      \begin{itemize}
        \item 休闲-境外游
        \item 娱乐-综艺
        \item 休闲-摄影
        \item ...
      \end{itemize}
      \end{block}
    \end{column}

    \pause

    \begin{column}{0.333\textwidth}
    \begin{block}{User3}
      \begin{itemize}
        \item 体育-足球
        \item 体育-NBA
        \item 游戏动漫
        \item ...
      \end{itemize}
      \end{block}
    \end{column}
  \end{columns}
\end{example}

\pause

用户兴趣是多维度的 $\Rightarrow$ 用户标签集合 \\
标签是根据business预先设定的 

\pause

\begin{block}{}
我们实际要解决的问题:如何给用户加\textbf{对应的标签}?
\end{block}
\end{frame}

\section{问题与求解}

\begin{frame}
\frametitle{Overview} % Table of contents slide, comment this block out to remove it
\tableofcontents[currentsection] % Throughout your presentation, if you choose to use \section{} and \subsection{} commands, these will automatically be printed on this slide as an overview of your presentation
\end{frame}

\subsection{问题抽象}

\begin{frame}
\frametitle{问题抽象}
To infer a \textbf{vector-valued} function $g : \mathcal{X} \rightarrow \mathcal{L}$ from a data set
$$\mathcal{D} = \{({\mathbf{x}}_1, {\mathbf{l}}_1), \cdots, ({\mathbf{x}}_1, {\mathbf{l}}_1)\} \in (\mathcal{X} \times \mathcal{L})^n$$
where $\mathbf{x} \in \mathcal{R}^d$ and $\mathbf{l} \in {\{+1,-1\}}^L$.
\end{frame}

\begin{frame}
\frametitle{Naive方案}
\begin{block}{}
To produce the vector-valued function $g : \mathcal{X} \rightarrow \mathcal{L}$, let's train a vector of \textit{binary} classifiers, each predicts the corresponding label.
\end{block}

\begin{block}{}
\begin{itemize}
\item + reuses existing techniques
\item + easy validation
\item - inefficient
\item - not (\textit{economically}) scalable
\end{itemize}
\end{block}
\end{frame}


\subsection{Preliminary: Boosting}

\begin{frame}
\frametitle{Preliminary: Boosting}
\begin{itemize}
\item Originally for binary classification
\item A sequence of learners each corrects the mis-classification by preceding ones
\item by training against a (\textit{re-})weighted sample set:
\end{itemize}
\end{frame}

\begin{frame}
\frametitle{Adaptive Boosting}

\begin{algorithm}[H]
\SetKwInOut{Input}{input}\SetKwInOut{Output}{output}
 \Input{$\mathcal{D} = (x_i, y_i) \in \{({\mathcal{R}}^n, \{-1, +1\}\}$, $T$, $\mathcal{B}$}
 \Output{$\mathbf{H} = \Sigma_{t=1}^T{{\alpha}_t {\mathbf{B}}_t}$}
 \Begin{
   ${\mathbf{D}}_1(i) = 1/m$\;
   \For{$t \leftarrow 1, \cdots, T$}{
    ${\mathbf{B}}_t \leftarrow \mathcal{B}(\mathcal{D}, {\mathbf{D}}_t)$\;
    Get hypothesis set: $h_t \leftarrow \mathbf{B}(\mathcal{D})$\;
    Get the base coefficient ${\alpha}_t$\;
    Update the weights: 
    $${{\mathbf{D}}_{t+1}}(i) = {{{\mathbf{D}}_{t}}(i) \exp(- {\alpha}_t y_i h_t(x_i))} / Z_t $$
    where $Z_t$ is a normalization factor\;
   }
 }
\end{algorithm}

\end{frame}

\subsection{Multi-label boosting}

\begin{frame}
\frametitle{Multi-Label Boosting: An efficient solution}
\begin{block}{The decomposition}
  Decompose the general vector-valued function $\mathfrak{g}$ into a \textit{label independent} scalar function to \textbf{split the input space} and a \textit{feature independent} vector-valued function to \textbf{cast the labels}.
\end{block}

\begin{block}{Boosting}
The weights $\mathbf{D}$ in AdaBoost are extended to each sample-label pair:
$${\Sigma}_{i,l} D(i, l) = 1$$
\end{block}

\end{frame}

\begin{frame}
\frametitle{AdaBoost.MH}
To produce a vector-valued discriminant function $\mathbf{f}^{(T)} : \mathcal{R}^d \rightarrow \mathcal{R}^K$ with a small Hamming loss $\hat{R}_H (\mathbf{f}^{(T)}, \mathbf{D})$ by minimizing the \textit{weighted multi-class exponential margin-based error}
$$\hat{R}_{EXP}(\mathbf{f}^{(T)}, \mathbf{D}) = \frac{1}{mK} {{\Sigma}_{i=1}^m} {{\Sigma}_{l=1}^K} {w_{i,l}\exp(- f_l^{(T)} y_{i,l})}$$

\begin{block}{}
\textbf{MH} means \textbf{M}ulti-class with \textbf{H}amming loss.
\end{block}

\begin{block}{}
Using \textit{decomposed} base learner.
\end{block}
\end{frame}

\begin{frame}
\frametitle{AdaBoost.MH}
\begin{algorithm}[H]
\SetKwInOut{Input}{input}\SetKwInOut{Output}{output}
 \Input{$\mathcal{D} = (x_i, y_i) \in \{({\mathcal{R}}^n, \{-1, +1\}^K\}$, $T$, $\mathcal{B}$}
 \Output{$\mathbf{H} = \Sigma_{t=1}^T{{\alpha}_t {\mathbf{v}}_t} {\varphi_t}$}
 \Begin{
   ${\mathbf{D}}_1(i, l) = \frac{1}{mK}$\;
   \For{$t \leftarrow 1, \cdots, T$}{
    $(\mathbf{v}_t, \varphi_t, \gamma_t) \leftarrow \mathcal{B}(\mathcal{D}, \mathbf{D}_t)$\;
    Get hypothesis set: $h_t \leftarrow \mathbf{B}(\mathcal{D})$\;
    Get the base coefficient ${\alpha}_t \leftarrow \frac{1}{2} \exp \frac{1+\gamma}{1-\gamma} $\;
    Update the weights: 
    $${{\mathbf{D}}_{t+1}}(i,l) = {{{\mathbf{D}}_{t}}(i,l) \exp(- {\alpha}_t y_{i,l} h_t(x_i))} / Z_t $$
    where $Z_t$ is a normalization factor\;
   }
 }
\end{algorithm}
\end{frame}

\begin{frame}
\frametitle{The base learner for AdaBoost.MH}
\begin{itemize}
\item How to get $\varphi$, i.e split the input space?
\item How to cast the labels, i.e produce the vector value using  $\varphi$?
\item How to calculate the base learner factor $\alpha$?
\end{itemize}
\end{frame}

\begin{frame}
\frametitle{Base learning}
\begin{itemize}
\item The \textit{weighted per-class error rate}
$$\mu_{l-} = \sum_{i=1}^{n}w_{i,j}\mathfrak{I}\{\varphi(\mathbf{x}_i) \neq y_{i,l}\}$$
\item The \textit{weighted per-class correct classification rate}
$$\mu_{l+} = \sum_{i=1}^{n}w_{i,j}\mathfrak{I}\{\varphi(\mathbf{x}_i) = y_{i,l}\}$$
\end{itemize}
\end{frame}

\begin{frame}
\frametitle{Base learning}
\begin{itemize}
\item The \textit{class-wise edge} with $\varphi(\bullet) \in \{\pm 1\}$ and $y_{i,l} \in \{\pm 1\}$
$$\gamma_l = v_l (\mu_{l+} - \mu_{l-}) = \sum_{i=1}^{n} w_{i,l} v_l \varphi(\mathbf{x}_i) y_{i,l}$$
\item The full multi-class edge of the classifier
$$\gamma = \gamma(\mathbf{v}, \varphi, \mathbf{W}) = \sum_{l=1}^{K} \gamma_l = \sum_{l=1}^{K} v_l(\mu_{l+} - \mu_{l-})$$
$$= \sum_{i=1}^{n} \sum_{l=1}^{K} w_{i,l} v_l \varphi(\mathbf{x}_i) y_{i,l} $$
\end{itemize}
\end{frame}

\begin{frame}
\frametitle{Base learning}
\begin{block}{Given $\varphi(\bullet)$, the votes $\mathbf{v}$}
which maximize the full edge is
\[v_l = \left\{
  \begin{array}{l l}
  1 \quad \quad \text{   if } \mu_{l+} > \mu_{l-} \\
  -1 \quad \text{ otherwise }
  \end{array} \right. \]
\end{block}
\begin{block}{$\varphi(\bullet)$ is determined by maximazing the full edge $\gamma$}
\begin{itemize}
\item Decision Stump: finding the best split cut
\item Hamming tree: node split with the max edge improvement
\item Generalized binary classifier.
\end{itemize}
\end{block}
\end{frame}

\begin{frame}
\frametitle{Generalized binary classifier as a base learner}
TODO: 介绍如何使用GBC作为base learner \\
目标函数是什么,为什么与 edge-maximization 是一致的
\end{frame}

\begin{frame}
\frametitle{The base learner factor}
\begin{block}{The EXP loss:}
$$Z(\mathbf{h}, \mathbf{W}) = \sum_{i=1}^{n} \sum_{l=1}^{K} w_{i,l} \exp(-\alpha v_l \varphi(\mathbf{x}_i) y_{i,l}) $$
$$= \frac{e^\alpha + e^{-\alpha}}{2} - \frac{e^\alpha - e^{-\alpha}}{2} \sum_{l=1}^{K} v_l(\mu_{l+} - \mu_{l-})$$
\end{block}
\begin{block}{$\alpha$ minimizing the EXP loss:}
$$\alpha = \frac{1}{2} \log \frac{1+\gamma}{1-\gamma}$$
\end{block}
\end{frame}

%----------------------------------------------------------------------------------------

\section{Multiboost on Spark}

\subsection{实现方案}

\begin{frame}
\frametitle{MultiBoost on Spark}
\begin{itemize}
\item Base learner如何实现
\item Strong learner实现方式
\item https://github.com/BaiGang/spark\_multiboost
\item 'Fork and Star' are welcome!
\end{itemize}
\end{frame}

\subsection{实验与效果}

\begin{frame}
\frametitle{Experiments setup}
\begin{itemize}
\item 数据集来源 -- 内部、外部通用测试集
\item 集群环境 TODO
\end{itemize}
\end{frame}

\begin{frame}
\frametitle{运行时间}
TODO: 对应数据集的训练时间长度
\end{frame}

\begin{frame}
\frametitle{效果}
\begin{itemize}
\item 对应数据集的训练和测试error,有图
\end{itemize}
\end{frame}


\subsection{后续工作}

\begin{frame}
\frametitle{MultiBoost on Spark}
\begin{itemize}
\item Base learner使用二分类,与edge-maximizing目标,需要数学证明
\item Hamming tree作为base learner
\item 现有时间基于Spark 1.1.x, 应用新的spark ml interface
\item 'Fork and Star' are welcome!
\end{itemize}
\end{frame}

\end{document} 
